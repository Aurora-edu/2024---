% !TeX program = xelatex
\documentclass{ctexart}
\usepackage[table]{xcolor}
\usepackage{template_by_mny}
\usepackage{float} 
\usepackage{listings}
\usepackage[figuresright]{rotating}
\lstset{basicstyle=\ttfamily, breaklines=true, frame=single}

\title{偏振实验报告}
\class{物理 32/物理 31}
\name{冯家琦/周方远}
\id{2023011338/2023011263}

\begin{document}

\maketitle

\section{引言}
本报告记录了使用偏振和3D电影技术套件进行的实验。这些实验旨在探索偏振原理、3D成像技术及其应用。
\section{实验原理}
\section{实验仪器}
\section{实验步骤}
\subsection{Part1:偏振实验}

\subsubsection{实验1:初步实验}
\begin{enumerate}
    \item 在激光器前放置两个偏振片,并将它们的透射轴调整至垂直。现在在两个交叉的偏振片之间拿着第三个偏振片(无支架的薄膜)并旋转它。你观察到了什么?为什么?
\end{enumerate}

\subsubsection{实验2:马吕斯定律}
\begin{enumerate}
    \item 在激光器前放置一个偏振片。将其定位,使得透过的光亮度最大化。现在在光束路径中放置第二个偏振片,并将光电探测器放在它后面。以你选择的增量大小旋转第二个偏振片,并记录光电探测器上的电压值。
\end{enumerate}

\subsubsection{实验3:测量激光的偏振状态}
\begin{enumerate}
    \item 在激光器前放置一个偏振片。光电探测器再次放置在偏振片后面。现在以你选择的增量大小旋转偏振片,并测量光电探测器上的电压。
\end{enumerate}

\subsubsection{实验4:确定 $\lambda/4$ 板的取向}
\begin{enumerate}
    \item 将板放置在两个垂直的偏振片之间,并寻找透射最大值以确定 $\lambda/4$ 板的 45° 取向。
\end{enumerate}

\subsubsection{实验5:$\lambda/4$ 板对绿光的行为}
\begin{enumerate}
    \item 首先在激光器前放置一个偏振片,然后在其前面放置 $\lambda/4$ 板。$\lambda/4$ 板应相对于偏振片调整至 45°。现在证明绿激光在通过 $\lambda/4$ 板后变为圆偏振光。
\end{enumerate}

\subsubsection{实验6:糖量测定}
\begin{enumerate}
    \item 通过测量线性偏振光通过糖溶液时的偏振旋转角度,确定比例常数。
\end{enumerate}

\subsubsection{实验7:确定软饮料中的糖含量}
\begin{enumerate}
    \item 通过测量偏振旋转来确定软饮料中的糖含量。
\end{enumerate}

\subsubsection{实验8:“轻”或“零”变体软饮料}
\begin{enumerate}
    \item 预计在“轻”或“零”变体软饮料中不会有偏振旋转。
\end{enumerate}

\subsubsection{实验9:应力双折射}
\begin{enumerate}
    \item 使用光学塑料盒观察应力双折射。
\end{enumerate}

\subsubsection{实验10:检查其他透明物体的应力双折射}
\begin{enumerate}
    \item 检查其他透明物体/材料的应力双折射,如气泡包装和眼镜。
\end{enumerate}

\end{document}