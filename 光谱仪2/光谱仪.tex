% !TeX program = xelatex
\documentclass{ctexart}
\usepackage[table]{xcolor}
\usepackage{template_by_mny}
\usepackage{float} 
\usepackage{listings}
\lstset{basicstyle=\ttfamily, breaklines=true, frame=single}

\title{光谱仪教学实验报告}
\class{物理 32/物理 31}
\name{冯家琦/周方远}
\id{2023011338/2023011263}

\begin{document}
\maketitle

\begin{abstract}
本实验使用EDU-SPEB2光谱仪教学套件,完成了LED光谱观测和已知谱线的棱镜折射率测量等实验。通过实验加深了对光谱仪工作原理的理解,掌握了光谱测量的基本方法。
\end{abstract}

\section{实验原理}

光谱仪是一种用于分析光谱的光学仪器。其基本原理包括:

\subsection{三棱镜折射}
三棱镜折射偏转角与折射率关系公式:
\[n=\frac{\sin\left(\frac{\gamma+60^\circ}{2}\right)}{\sin 30^\circ}\]

其中$n$是折射率,$\gamma$是偏转角。不同波长的光折射率不同,因此通过三棱镜后的偏转角也不同,不同波长的光会被偏转到不同方向。
\subsection{光谱仪的组成}
一个基本的光谱仪包含以下部件:
\begin{itemize}
    \item 入射狭缝:控制入射光束宽度
    \item 准直系统:将发散光束准直
    \item 分光元件:三棱镜
    \item 聚焦系统:将分光后的光聚焦
    \item 观测屏/探测器:用于观测光谱
\end{itemize}

\section{实验仪器及安装}
\subsection{实验仪器}
\begin{itemize}
    \item EDU-SPEB2光谱仪教学套件
    \item 可调节狭缝
    \item 三棱镜
    \item LED光源及其USB供电支架
    \item 双凸透镜(f=50mm和100mm)
    \item 观测屏
    \item 其他光学元件支架和固定件
\end{itemize}
\subsection{仪器安装}
\subsubsection{光源安装}
三棱镜光谱仪光源部分与上周光栅光谱仪的光源部分相同,因而可以保留上周光栅光谱仪的光源,并将其余部分拆除。

\subsubsection{三棱镜光谱仪安装}
在狭缝后约 20cm 处放置三棱镜并固定。

在三棱镜与狭缝正中间放置 100mm 透镜,并确保狭缝、透镜、棱镜三者在同一光路中。

在三棱镜折射白光的出射处放置观察屏,并放置黑色挡板阻止LED光直射观察屏。转动三棱镜使得可见光光谱正好打在光屏上。

在狭缝后安装绿色滤光片,这样观察屏上仅剩下很窄的光谱,便于对焦。调整 100mm 透镜的位置和光屏的位置,并微调三棱镜的角度,使得观察屏上呈现清晰的像。对好焦后将绿色滤光片取下。
\section{实验步骤}

\subsection{实验一:LED光谱观测}
**
1. 将LED安装在USB供电支架上并通电。LED的光谱如图,可以看到有两个尖峰,并且在蓝色和绿色之间波长的光非常弱,几乎消失。


2. 经过观察如图

与理论相符,可以看到蓝色和绿色之间的光强非常弱。

\subsection{实验二:已知光谱线的棱镜折射率测量}
**

\section{实验思考}

\subsection{LED光谱特征}
LED发出的是连续光谱,但在特定波长范围内强度较大。这反映了LED的发光机理是由电子在半导体能带间跃迁产生的。

\subsection{三棱镜光谱测量}
**

\section{总结}

通过本次实验,我们:
\begin{itemize}
    \item 理解了光谱仪的基本工作原理
    \item 掌握了调节光谱仪的方法
    \item 学会了测量三棱镜的折射率
    \item 认识到了不同光源的光谱特征
\end{itemize}

实验过程中发现光路调节较为关键,需要仔细进行准直和聚焦。建议在后续实验中可以增加不同光源的对比分析。

\end{document}