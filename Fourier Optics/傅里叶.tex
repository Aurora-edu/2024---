% !TeX program = xelatex
\documentclass{ctexart}
\usepackage{template_by_mny}

\title{傅里叶光学实验报告}
\class{物理 32}
\name{冯家琦/周方远}
\id{2023011338}

\begin{document}
\maketitle

\begin{abstract}
  本实验旨在通过搭建傅里叶光学系统,观察并理解光波的衍射、成像、滤波等基本现象,
  并探究傅里叶光学在光学成像中的应用。
\end{abstract}

\section{实验原理}
傅里叶光学是研究光波经过光学系统后的衍射和成像规律的科学。
本实验采用4f光学系统,利用透镜对光波进行变换,在傅里叶平面上观察到光波的频谱信息,并通过滤波等手段实现对图像的处理。

\section{实验仪器及实验步骤}
  仪器:* LED光源
  * 准直透镜
  * 物镜
  * 套筒透镜
  * 相机
  * Target
  * Condenser lens
  * Aperture Iris
  * Field Iris
  * Projection lens
  * 观察屏
  * 分束器
  * 滤光片
  * Mask
  \subsection{调节望远镜}

\section{实验步骤}
1. 搭建4f光学系统,按照文档中提供的步骤进行调节,确保各个光学元件的高度、位置和角度正确。

2. 使用相机观察Target上的图案,并通过调节焦距和曝光时间获得清晰的图像。

3. 调节分束器和Projection lens,将傅里叶平面上的图像投射到观察屏上。

4. 进行以下实验内容:
    * \textbf{开普勒望远镜成像}:观察Field lens和Condenser lens、Condenser lens和Objective lens组合后的成像效果。
    
    * \textbf{观察傅里叶面}:观察Target上不同图案的傅里叶变换图像。
    
    * \textbf{Field Iris和Aperture Iris的作用}:分析光阑对成像、傅里叶面和物体照明的影响。
    
    * \textbf{Projection lens成像}:观察图像从清晰到模糊再到傅里叶平面的变化过程。
    
    * \textbf{光学滤波}:使用狭缝和Mask对傅里叶图像进行滤波,观察滤波对最终成像的影响。
    
    * \textbf{巴比内原理}:观察互补结构的傅里叶变换图像。
    
    * \textbf{图像编辑}:通过调节光阑和Mask实现柔焦和暗场成像。
    
    * \textbf{衍射极限分辨率}:验证阿贝衍射极限定律。

\section{实验结果}
\subsection{开普勒望远镜成像}
* Field lens和Condenser lens组合后,图像被放大并成倒立。
* Condenser lens和Objective lens组合后,图像被放大并成正立。
\subsection{观察傅里叶面}
* Target上不同图案的傅里叶变换图像呈现出不同的衍射斑。
\subsection{Field Iris和Aperture Iris的作用}
* Field Iris控制物体照明区域,影响图像亮度。
* Aperture Iris控制成像系统孔径,影响图像分辨率和衍射斑形状。
\subsection{Projection lens成像}
* 图像从清晰到模糊再到傅里叶平面的变化过程体现了光学成像的原理。
\subsection{光学滤波}
* 通过滤波可以突出图像的特定频率成分,实现图像增强或去噪。
\subsection{巴比内原理}
* 互补结构的傅里叶变换图像相同,验证了巴比内原理。
\subsection{图像编辑}
* 柔焦可以突出图像的低频成分,暗场成像可以突出图像的高频成分。
\subsection{衍射极限分辨率}
* 实验验证了阿贝衍射极限定律,只有当至少第一个衍射阶次通过成像系统时,才能形成清晰的图像。

\section{实验思考}
* 如何进一步提高光学系统的分辨率?
* 如何利用傅里叶光学进行图像重建?
* 傅里叶光学在哪些领域具有应用前景?

\section{总结}
\begin{itemize}
  \item 本实验通过搭建傅里叶光学系统,观察并理解了光波的衍射、成像、滤波等基本现象,
  并探究了傅里叶光学在光学成像中的应用。实验结果表明,傅里叶光学在图像处理、光学测量等领域具有广泛的应用前景。

\end{itemize}
\end{document}