% !TeX program = xelatex
\documentclass{ctexart}
\usepackage[table]{xcolor}
\usepackage{template_by_mny}
\usepackage{float} 
\usepackage{listings}
\lstset{basicstyle=\ttfamily, breaklines=true, frame=single}

\title{光谱仪教学实验报告}
\class{物理 32/物理 31}
\name{冯家琦/周方远}
\id{2023011338/2023011263}

\begin{document}
\maketitle

\begin{abstract}
本实验使用EDU-SPEBCT1扩展光谱仪教学套件,完成了绿光波长测量、样品吸收率光谱图等实验。通过实验加深了对光谱仪工作原理的理解,掌握了光谱测量的进阶方法。
\end{abstract}

\section{实验原理}

光谱仪是一种用于分析光谱的光学仪器。其基本原理包括:

\subsection{光谱仪的组成}
一个基本的光谱仪包含以下部件:

\section{实验仪器及安装}
\subsection{实验仪器}
\begin{itemize}
    \item EDU-SPEB2光谱仪教学套件
    \item 可调节狭缝
    \item 三棱镜
    \item LED光源及其USB供电支架
    \item 双凸透镜(f=50mm和100mm)
    \item 观测屏
    \item 其他光学元件支架和固定件
\end{itemize}
\subsection{仪器安装}
\section{实验步骤}
\subsection{确定波长}

\subsection{参考光谱}
**

\subsection{样品光谱}
**

\subsection{数据分析:吸收谱线}
**

\section{实验思考}

\section{总结}
**

\end{document}